\input{template_settings/ch_preamble} % лучше не редактировать / please, keep unmodified
\DeclareGraphicsExtensions{.pdf, .png, .jpg}
\graphicspath{{./my_folder/images/}}

\usepackage{tikz}
\usetikzlibrary{arrows,automata,positioning}
\usepackage{float}

\usepackage{amssymb}


\usepackage{CJKutf8}
\usepackage{pdfpages}
\usepackage{CJK,CJKspace,CJKpunct}
\pdfmapline{=unisong@Unicode@ <ipam.ttf}
\usepackage{minted}

\usepackage{lmodern}

% вертикальное выравнивание колонок в таблице
\renewcommand\tabularxcolumn[1]{m{#1}}

\newcommand{\commands}{C:/Users/Ruminat/GoogleDrive/TeX/commands}
\newcommand{\import}[1]{\input{\commands/#1}}
\import{old/textCommands} % текстовые команды

\newcommand{\SQLcode}[1]{\inputminted[tabsize=2, mathescape, fontsize=\footnotesize]{sql}{tasks/code/#1}}
\newcommand{\SQLplan}[1]{\inputminted[tabsize=2, mathescape, fontsize=\fontsize{7}{7}\selectfont]{text}{tasks/plans/#1}}

\setcounter{docType}{1} % лучше не редактировать / please, keep unmodified

%%%% Настройки автора / Author settings
%% 
\input{my_folder/my_settings} % добавляем свои команды / update your commands

\begin{document} % начало документа
  \input{template_settings/common/renames} % Заполнить сведения, 
  % \input{my_folder/title}          % Титульный лист

  % title
  \thispagestyle{empty}%
  \includepdf[fitpaper=true, pages=-, pagecommand={}]{title.pdf}

  \input{my_folder/contents}             % Оглавление

  %% Начало основной части
  %!TeX root = ../My_thesis.tex
\chapter*{Введение} % * не проставляет номер
\addcontentsline{toc}{chapter}{Введение} % вносим в содержание

Оптимизация запросов "--- важная часть администрирования баз данных.
От скорости выполнения зависит скорость работы всего приложения (если база данных используется для какого-либо приложения).
К примеру, в интернет-магазине, чем дольше пользователь ждёт, пока загрузятся данные, тем выше вероятность, что он покинет сайт в пользу конкурента.
Оптимизация так же помогает избежать траты на излишнюю покупка железа, которое приобретают, чтобы ускорить работу базы данных.
Поэтому крайне важно научиться анализировать запросы на возможность их оптимизации.

Целью данной работы является практика оптимизирования запросов в СУБД Oracle.
Конкретно "--- оптимизироваться будет стоимость запросов (cost), которая вычисляется по формуле~\eqref{eq_cost}~\cite{ExplainPlan}.
\begin{equation}\label{eq_cost}%
  \operatorname{cost} = \frac{
    (\operatorname{\#SRds} \cdot \operatorname{sreadtim})
      + (\operatorname{\#MRds} \cdot \operatorname{mreadtim})
      + (\operatorname{\#cpuCycles} / \operatorname{cpuSpeed})
  }{
    \operatorname{sreadtim}
  },
\end{equation}
где
\begin{itemize}%
  \item #SRDs "--- количество одноблочных чтений.
  \item sreadtim "--- время на одно одноблочное чтение.
  \item #MRDs "--- количество многоблочных чтений.
  \item mreadtim "--- время одно на многоблочное чтение.
  \item \#CPUCycles "--- количество циклов CPU. Включает в себя стоимость CPU-обработки (чистая стоимость CPU) и стоимость извлечения данных (стоимость buffer cache get в CPU).
  \item cpuSpeed "--- количество CPU-циклов в секунду.
\end{itemize}

Порядок выполнения работы будет следующим:
\begin{enumerate}[1.]%
  \item Написать SQL-запрос для каждой задачи.
  \item Проанализировать планы выполнения, статистики, а также сами запросы на возможные варианты оптимизаций.
  \item Использовать как минимум 2 оптимизации на каждый запрос (причём одна оптимизация может быть использована не более, чем дважды за всю работу).
  \item Оптимизация может считаться успешной при уменьшении стоимости~(cost).
\end{enumerate}
         % Введение
  %!TeX root = ../My_thesis.tex


\chapter*{Описание окружения} % * не проставляет номер
\addcontentsline{toc}{chapter}{Описание окружения} % вносим в содержание


Используется база данных Oracle версии \texttt{Oracle Database 21c Express Edition Release 21.0.0.0.0 - Production Version 21.3.0.0.0}.
Схемы создавались в Portable Database~\cite{PDB} \texttt{xepdb1}, поэтому строки подключения имели вид: \texttt{CONNECT hr/PASSWORD@xepdb1}.

Стандартные схемы Oracle (\texttt{HR}, \texttt{OE} и~т.\,д.) были взяты из оффициального репозитория Oracle~\cite{SampleSchemas}.

Схема \texttt{STUDENT} была взята с учебного курса «Программирование и оптимизация баз данных»~\cite{StudentSchema}.

Перед каждым выполнением SQL-запроса выполняется очистка Buffer Cache и Shared Pool с помощью первых двух команд ниже.
После чего выполняется запрос с выводом плана.

\begin{minted}[tabsize=2, mathescape, fontsize=\footnotesize]{sql}
alter system flush shared_pool;
alter system flush buffer_cache;
set autotrace on
@task-/*НОМЕР-ЗАДАНИЯ*/.sql
set autotrace off
\end{minted}             % Глава 1
  % \ContinueChapterBegin % размещать главы <<подряд>> 
  % .---|||___|||--- S E C T I O N ---|||___|||---. %
\chapter{Задача «Процент мужщин/женщин в заданную дату (OE)»}
% .---|||___|||--- S E C T I O N ---|||___|||---. %


% .---|||___|||--- S U B S E C T I O N ---|||___|||---. %
\section{Описание задачи}
% .---|||___|||--- S U B S E C T I O N ---|||___|||---. %


Используются таблицы схемы \texttt{OE}. Вывести процентное соотношение мужчин и женщин, разместивших заказы в заданную дату. Если один и тот же человек разместил несколько заказов в заданную дату, он должен быть учтён только один раз. В результате должно быть три столбца: дата, процент мужчин и процент женщин.


% .---|||___|||--- S U B S E C T I O N ---|||___|||---. %
\section{Составленный запрос}
% .---|||___|||--- S U B S E C T I O N ---|||___|||---. %


\SQLcode{task-1.sql}
\begin{algorithm}[H]
  \caption{Запрос для задачи №1}
  \label{code-task-1}
\end{algorithm}

\begin{samepage}%
  \begin{minted}[tabsize=2, mathescape, fontsize=\tiny]{text}%
Date        Males (%) Females (%)
---------- ---------- -----------
29-06-2007         75          25
  \end{minted}
  \begin{figure}[H]%
    \caption{Результат запроса для даты «29-06-2007»}
    \label{fig-task-1-output}
  \end{figure}
\end{samepage}


\SQLplan{1-no-op.txt}
\begin{figure}[H]%
  \caption{План выполнения запроса}
  \label{fig-task-1-plan}
\end{figure}


% .---|||___|||--- S U B S E C T I O N ---|||___|||---. %
\section{Первая оптимизация (function-based индекс)}
% .---|||___|||--- S U B S E C T I O N ---|||___|||---. %


Обратим внимание на то, что в исходном запросе используется сравнение даты, использующее функцию \texttt{TRUNC}.
Создадим function-based индекс на эту функцию на столбец \texttt{order\_date}.

\begin{minted}[tabsize=2, mathescape, fontsize=\tiny]{sql}%
CREATE INDEX orders_order_date_fnidx ON oe.orders (TRUNC(order_date, 'dd'));
\end{minted}
\begin{figure}[H]%
  \caption{Создание function-based индекса}
  \label{fig-task-1-plan}
\end{figure}

\SQLplan{1-function-based.txt}
\begin{figure}[H]%
  \caption{План выполнения оптимизированного запроса}
  \label{fig-task-1-function-based-plan}
\end{figure}

Как можно увидеть по плану, cost уменьшился на $\sim 14$\% (56~$\to$~48), поменялись также статистики (не в лучшую сторону):
\begin{itemize}%
  \item \texttt{recursive calls} "--- 2732~$\to$~2992,
  \item \texttt{db block gets} "--- 0~$\to$~5,
  \item \texttt{consistent gets} "--- 3905~$\to$~4135,
  \item \texttt{physical reads} "--- 478~$\to$~491,
  \item \texttt{redo size} "--- 0~$\to$~852,
  \item \texttt{sorts (memory)} "--- 248~$\to$~271.
\end{itemize}


% .---|||___|||--- S U B S E C T I O N ---|||___|||---. %
\section{Вторая оптимизация (bitmap индекс)}
% .---|||___|||--- S U B S E C T I O N ---|||___|||---. %


Обратим также внимание на то, что столбец \texttt{gender} таблицы \texttt{customers} имеет всего 2 значения "--- «F» и «M», из чего напрашивается желание создать bitmap-индекс на этот столбец.

\begin{minted}[tabsize=2, mathescape, fontsize=\tiny]{sql}%
CREATE BITMAP INDEX customers_gender_btmidx ON oe.customers (gender);
\end{minted}
\begin{figure}[H]%
  \caption{Создание bitmap индекса}
  \label{fig-task-1-bitmap-plan}
\end{figure}

\SQLplan{1-bitmap.txt}
\begin{figure}[H]%
  \caption{План выполнения оптимизированного запроса}
  \label{fig-task-1-bitmap-plan}
\end{figure}

В плане видим, что стоимость (cost) уменьшилась на $\sim 8$\% (56~$\to$~51). Посмотрим, как поменялись статистики (в основном, в худшую сторону):
\begin{itemize}%
  \item \texttt{recursive calls} "--- 2732~$\to$~2988,
  \item \texttt{consistent gets} "--- 3905~$\to$~4087,
  \item \texttt{physical reads} "--- 478~$\to$~432,
  \item \texttt{sorts (memory)} "--- 248~$\to$~274.
\end{itemize}

% .---|||___|||--- S E C T I O N ---|||___|||---. %
\chapter{Задача «Имена сотрудников, встречающиеся более 2-ух раз (HR)»}
% .---|||___|||--- S E C T I O N ---|||___|||---. %


% .---|||___|||--- S U B S E C T I O N ---|||___|||---. %
\section{Описание задачи}
% .---|||___|||--- S U B S E C T I O N ---|||___|||---. %


Используются таблицы схемы \texttt{HR}. Вывести имена сотрудников, встречающиеся в таблице сотрудников не менее трех раз и не являющиеся именами руководителей подразделений компании или именами непосредственных руководителей кого-либо. В результате должны быть выведены только имена сотрудников, причём каждое "--- только один раз.


% .---|||___|||--- S U B S E C T I O N ---|||___|||---. %
\section{Составленный запрос}
% .---|||___|||--- S U B S E C T I O N ---|||___|||---. %


\SQLcode{task-2.sql}
\begin{algorithm}[H]
  \caption{Запрос для задачи №2}
  \label{code-task-2}
\end{algorithm}


\begin{samepage}%
  \begin{minted}[tabsize=2, mathescape, fontsize=\tiny]{text}%
FIRST_NAME
----------
David
Peter
  \end{minted}
  \begin{figure}[H]%
    \caption{Результат запроса}
    \label{fig-task-2-output}
  \end{figure}
\end{samepage}


\SQLplan{2-no-op.txt}
\begin{figure}[H]%
  \caption{План выполнения запроса}
  \label{fig-task-2-plan}
\end{figure}


% .---|||___|||--- S U B S E C T I O N ---|||___|||---. %
\section{Первая оптимизация (переписывание запроса)}
% .---|||___|||--- S U B S E C T I O N ---|||___|||---. %


Попробуем использовать вместо \texttt{MINUS} оператор \texttt{NOT EXISTS} с подзапросом (\firef{code-task-2-rewritten}).

\SQLcode{task-2-rewritten.sql}
\begin{algorithm}[H]
  \caption{Переписанный запрос}
  \label{code-task-2-rewritten}
\end{algorithm}


\SQLplan{2-rewrite.txt}
\begin{figure}[H]%
  \caption{План выполнения переписанного запроса}
  \label{fig-task-2-rewrite-plan}
\end{figure}

Смотрим, как изменился план. Стоимость (cost) уменьшилась довольно незначительно "--- на $\sim 2$\% (107~$\to$~105).
Статистики же не изменились никоим образом.


% .---|||___|||--- S U B S E C T I O N ---|||___|||---. %
\section{Вторая оптимизация (материализованное представление)}
% .---|||___|||--- S U B S E C T I O N ---|||___|||---. %


Попробуем более масштабную оптимизацию "--- создадим материализованное представление.
Оно может очень сильно ускорить запрос.
Недостатком, тем не менее, является то, что данные в представлении нужно поддерживать (при изменении данных в исходной таблице), однако таблица \texttt{employees} маловероятно будет меняться часто (так как сотрудники в компании не приходят и уходят каждую минуту), может быть и такое, что таблица может не меняться днями. Поэтому использование материализованного представления здесь вполне оправдано.

Создание материализованного представления показано на~\firef{jojo}.

\begin{minted}[tabsize=2, mathescape, fontsize=\tiny]{sql}%
CREATE MATERIALIZED VIEW employees_names_mview
  BUILD IMMEDIATE
  REFRESH COMPLETE
  AS /* ИСХОДНЫЙ ЗАПРОС */
\end{minted}
\begin{figure}[H]%
  \caption{Создание материализованного представления \texttt{employees\_names\_mview}}
  \label{jojo}
\end{figure}

\SQLplan{2-mview.txt}
\begin{figure}[H]%
  \caption{План выполнения материализованного представления}
  \label{fig-task-2-mview-plan}
\end{figure}

Глядя на план, можно увидеть, как сильно изменился cost "--- на $\sim 97$\% (107~$\to$~3).
Некоторые статистики, тем не менее, ухудшились (кроме \texttt{physical reads} "--- они улучшились):
\begin{itemize}%
  \item recursive calls "--- 526~$\to$~934,
  \item consistent gets "--- 1189~$\to$~1460,
  \item physical reads "--- 420~$\to$~96,
  \item bytes sent via SQL*Net to client "--- 2023~$\to$~2239,
  \item sorts (memory) "--- 68~$\to$~81.
\end{itemize}

% .---|||___|||--- S E C T I O N ---|||___|||---. %
\chapter{Задача «Вывод групп с отличниками и должниками (STUDENT)»}
% .---|||___|||--- S E C T I O N ---|||___|||---. %


% .---|||___|||--- S U B S E C T I O N ---|||___|||---. %
\section{Описание задачи}
% .---|||___|||--- S U B S E C T I O N ---|||___|||---. %


Используются таблицы схемы \texttt{STUDENT}. Создать запрос для получения информации о группах в виде, представленном в таблице\,\ref{tab-task-2-required-output}.
\begin{table}[H]%
  \centering%
  \caption{Требуемый вывод для задачи №3}
  \label{tab-task-2-required-output}
  \noindent\begin{tabularx}{\textwidth}{|C|C|C|C|C|}%
    \hline
    Группа & Кол-во студентов & Название специальности & Кол-во круглых отличников & Кол-во должников \\
    \hline
    121 &  &  &  & \\ 
    \hline
    122 &  &  &  & \\ 
    \hline
    ... &  &  &  & \\ 
    \hline
  \end{tabularx}
\end{table}%

При подсчёте отличников учесть, что пятёрка могла быть получена со второй попытки. Для всех учитывать только экзамены, предусмотренные учебным планом.


% .---|||___|||--- S U B S E C T I O N ---|||___|||---. %
\section{Составленный запрос}
% .---|||___|||--- S U B S E C T I O N ---|||___|||---. %


\SQLcode{task-3.sql}

\begin{algorithm}[H]
  \caption{Запрос для задачи №3}
  \label{code-task-3}
\end{algorithm}

\begin{minted}[tabsize=2, mathescape, fontsize=\tiny]{text}%
Группа  Кол-во студентов Кол-во студентов                    Кол-во круглых отличников Кол-во должников
------- ---------------- ----------------------------------- ------------------------- ----------------
121                    3 СИСТЕМЫ АВТОМАТИЧЕСКОГО УПРАВЛЕНИЯ                          1                1
122                    2 СИСТЕМЫ АВТОМАТИЧЕСКОГО УПРАВЛЕНИЯ                          1                0
123                    2 ЭКОНОМИКА ПРЕДПРИЯТИЙ                                       0                1
124                    2 ЭКОНОМИКА ПРЕДПРИЯТИЙ                                       0                0
\end{minted}

\begin{figure}[H]%
  \caption{Результат запроса}
  \label{fig-task-3-output}
\end{figure}


\SQLplan{3-no-op.txt}
\begin{figure}[H]%
  \caption{План выполнения запроса}
  \label{fig-task-3-plan}
\end{figure}


% .---|||___|||--- S U B S E C T I O N ---|||___|||---. %
\section{Первая оптимизация (bitmap индекс)}
% .---|||___|||--- S U B S E C T I O N ---|||___|||---. %


\SQLplan{3-group-num-index.txt}
\begin{figure}[H]%
  \caption{План выполнения оптимизированного запроса}
  \label{fig-task-4-b-tree-plan}
\end{figure}


% .---|||___|||--- S E C T I O N ---|||___|||---. %
\chapter{Задача «Клиент, сделавший покупку на максимальную сумму (SH)»}
% .---|||___|||--- S E C T I O N ---|||___|||---. %


% .---|||___|||--- S U B S E C T I O N ---|||___|||---. %
\section{Описание задачи}
% .---|||___|||--- S U B S E C T I O N ---|||___|||---. %


Используются таблицы схемы \texttt{SH}. Вывести фамилию и имя клиента, сделавшего покупки через интернет (\texttt{channel\_desc = "Internet"}) или партнёров (\texttt{channel\_desc = "Partners"}) не по акции (\texttt{promo\_category = "NO PROMOTION \#"}) на максимальную сумму в заданном году.


% .---|||___|||--- S U B S E C T I O N ---|||___|||---. %
\section{Составленный запрос}
% .---|||___|||--- S U B S E C T I O N ---|||___|||---. %


\SQLcode{task-4.sql}

\begin{algorithm}[H]
  \caption{Запрос для задачи №4}
  \label{code-task-4}
\end{algorithm}

\begin{minted}[tabsize=2, mathescape, fontsize=\tiny]{text}%
Surname                                  Name                      Spent
---------------------------------------- -------------------- ----------
Bakerman                                 Marvel                 56243,93
\end{minted}

\begin{figure}[H]%
  \caption{Результат запроса для года 1998}
  \label{fig-task-4-output}
\end{figure}


\SQLplan{4-no-op.txt}
\begin{figure}[H]%
  \caption{План выполнения запроса}
  \label{fig-task-4-plan}
\end{figure}


% .---|||___|||--- S U B S E C T I O N ---|||___|||---. %
\section{Первая оптимизация (индексно-организованная таблица)}
% .---|||___|||--- S U B S E C T I O N ---|||___|||---. %


В таблице \texttt{customers} довольно большое количество столбцов (больше 10), однако в нашем запросе используются лишь 3, отсюда появляется идея создать индексно-организованную таблицу \texttt{customer\_lookup}, содержащую эти столбцы. Это мы и сделаем (\firef{pekoooooora}).

\begin{minted}[tabsize=2, mathescape, fontsize=\tiny]{sql}%
CREATE TABLE customer_lookup (
  cust_id,
  cust_last_name,
  cust_first_name,
  CONSTRAINT customers_iot_pk PRIMARY KEY (cust_id)
)
ORGANIZATION INDEX
AS
  SELECT
    cust_id,
    cust_last_name,
    cust_first_name
  FROM sh.customers;
\end{minted}
\begin{figure}[H]%
  \caption{Создание индексно-организованной таблицы \texttt{customer\_lookup}}
  \label{pekoooooora}
\end{figure}

Как и в предыдущей задаче, нам нужно всего лишь заменить все вхождения \texttt{sh.customer} на \texttt{sh.customer\_lookup}.

\SQLplan{4-iot.txt}
\begin{figure}[H]%
  \caption{План выполнения оптимизированного запроса}
  \label{fig-task-4-iot-plan}
\end{figure}

Как можно заметить на плане, стоимость запроса существенно снизилась "--- на $\sim 40$\% (1059~$\to$~635).
Посмотрим на изменение статистик (многие статистики улучшились, причём некоторые "--- довольно ощутимо, "--- хотя некоторые, всё же ухудшились, особенно \texttt{consistent gets}):
\begin{itemize}%
  \item \texttt{recursive calls} "--- 2471~$\to$~2227,
  \item \texttt{db block gets} "--- 23~$\to$~0,
  \item \texttt{consistent gets} "--- 4264~$\to$~52689,
  \item \texttt{physical reads} "--- 610~$\to$~2054,
  \item \texttt{redo size} "--- 4052~$\to$~0,
  \item \texttt{bytes sent via SQL*Net to client} "--- 38650~$\to$~735,
  \item \texttt{bytes received via SQL*Net from client} "--- 536~$\to$~52,
  \item \texttt{SQL*Net roundtrips to/from client} "--- 46~$\to$~2,
  \item \texttt{sorts (memory)} "--- 246~$\to$~197,
  \item \texttt{rows processed} "--- 671~$\to$~1.
\end{itemize}


% .---|||___|||--- S U B S E C T I O N ---|||___|||---. %
\section{Вторая оптимизация (компизитный B-tree индекс)}
% .---|||___|||--- S U B S E C T I O N ---|||___|||---. %


Есть также другой путь оптимизации 3-х столбцов из предыдущей оптимизации "--- заметим, что все эти столбцы находятся внутри \texttt{GROUP BY}, что как бы намекает нам на создание композитного B-tree индекса (B-tree потому, что значения не имеют тенденции иметь много одинаковых значений). Этим мы и займёмся "--- создадим такой индекс (\firef{baqua}).

\begin{minted}[tabsize=2, mathescape, fontsize=\tiny]{sql}%
CREATE INDEX customers_cmpidx ON sh.customers (cust_id, cust_first_name, cust_last_name);
\end{minted}
\begin{figure}[H]%
  \caption{Создание композитного индекса}
  \label{baqua}
\end{figure}

\SQLplan{4-composite.txt}
\begin{figure}[H]%
  \caption{План выполнения оптимизированного запроса}
  \label{fig-task-4-composite-plan}
\end{figure}

Теперь посмотрим, как изменился cost. Улучшение поменьше, чем при первой оптимизации "--- на $\sim 34$\% (1059~$\to$~700), "--- однако плюсом этой оптимизации является то, что нам не нужно поддерживать отдельную таблицу \texttt{sh.customer\_lookup}, хотя и нужно поддерживать индекс, при этом прошлая оптимизация работает лишь для \texttt{sh.customer\_lookup}, текущая "--- для изначальной таблицы.
Посмотрим на статистики (здесь, по сравнению с прошлой оптимизацией, не произошло ухудшения статистик, однако и улучшения не столь значительны):
\begin{itemize}%
  \item \texttt{recursive calls} "--- 4230~$\to$~4217,
  \item \texttt{consistent gets} "--- 10016~$\to$~8701,
  \item \texttt{physical reads} "--- 3582~$\to$~2286,
  \item \texttt{sorts (memory)} "--- 374~$\to$~370.
\end{itemize}

% .---|||___|||--- S E C T I O N ---|||___|||---. %
\chapter{Задача «Список сотрудников по должностям и зарплатам (HR)»}
% .---|||___|||--- S E C T I O N ---|||___|||---. %


% .---|||___|||--- S U B S E C T I O N ---|||___|||---. %
\section{Описание задачи}
% .---|||___|||--- S U B S E C T I O N ---|||___|||---. %


Используются таблицы схемы \texttt{HR}. Одной командой \texttt{SELECT} вывести список сотрудников компании, имеющих коллег с таким же идентификатором должности и окладом. Если некоторый идентификатор должности и размер оклада имеет один единственный сотрудник, то сведения о нём в результат попадать не должны.

В результат вывести:
\begin{enumerate}[1.]%
  \item идентификатор должности;
  \item размер оклада;
  \item список фамилий сотрудников, имеющих данный идентификатор должности и данный оклад.
\end{enumerate}

Фамилии в списке должны быть:
\begin{enumerate}[a.]%
  \item упорядочены по алфавиту (по возрастанию),
  \item разделены символами \texttt{', '} («запятая» и «пробел»),
  \item перед первой фамилией не должно быть символов-разделителей,
  \item после последней фамилии символов-разделителей быть не должно.
\end{enumerate}

Результат упорядочить:
\begin{enumerate}[1.]%
  \item по размеру оклада (по убыванию),
  \item по идентификатору должности (по возрастанию).
\end{enumerate}


% .---|||___|||--- S U B S E C T I O N ---|||___|||---. %
\section{Составленный запрос}
% .---|||___|||--- S U B S E C T I O N ---|||___|||---. %


\SQLcode{task-5.sql}

\begin{algorithm}[H]
  \caption{Запрос для задачи №5}
  \label{code-task-5}
\end{algorithm}

\begin{minted}[tabsize=2, mathescape, fontsize=\tiny]{text}%
Job ID         Salary Surnames
---------- ---------- ------------------------------
AD_VP           17000 De Haan, Kochhar
SA_REP          10000 Bloom, King, Tucker
SA_REP           9500 Bernstein, Greene, Sully
SA_REP           9000 Hall, McEwen
SA_REP           8000 Olsen, Smith
SA_REP           7500 Cambrault, Doran
SA_REP           7000 Grant, Sewall, Tuvault
SA_REP           6200 Banda, Johnson
IT_PROG          4800 Austin, Pataballa
ST_CLERK         3300 Bissot, Mallin
SH_CLERK         3200 McCain, Taylor

Job ID         Salary Surnames
---------- ---------- ------------------------------
ST_CLERK         3200 Nayer, Stiles
SH_CLERK         3100 Fleaur, Walsh
SH_CLERK         3000 Cabrio, Feeney
SH_CLERK         2800 Geoni, Jones
ST_CLERK         2700 Mikkilineni, Seo
SH_CLERK         2600 Grant, OConnell
SH_CLERK         2500 Perkins, Sullivan
ST_CLERK         2500 Marlow, Patel, Vargas
ST_CLERK         2400 Gee, Landry
ST_CLERK         2200 Markle, Philtanker
\end{minted}
\begin{figure}[H]%
  \caption{Результат запроса}
  \label{fig-task-5-output}
\end{figure}


\SQLplan{5-no-op.txt}
\begin{figure}[H]%
  \caption{План выполнения запроса}
  \label{fig-task-5-plan}
\end{figure}


% .---|||___|||--- S U B S E C T I O N ---|||___|||---. %
\section{Первая оптимизация (материализованное представление)}
% .---|||___|||--- S U B S E C T I O N ---|||___|||---. %


Как мы уже обсуждали ранее, материализованное представление "--- довольно хороший выбор для таблицы \texttt{employees}, текущий запрос "--- не исключение (он возвращает не так много строк, поэтому представление будет небольшим). Создадим материализованное представление с помощью скрипта на \firef{jija}.

\begin{minted}[tabsize=2, mathescape, fontsize=\tiny]{sql}%
CREATE MATERIALIZED VIEW employees_job_salary_mview
  BUILD IMMEDIATE
  REFRESH COMPLETE
  AS /* ИСХОДНЫЙ ЗАПРОС */
\end{minted}
\begin{figure}[H]%
  \caption{Создание материализованного представления \texttt{employees\_job\_salary\_mview}}
  \label{jija}
\end{figure}

\SQLplan{5-mview.txt}
\begin{figure}[H]%
  \caption{План выполнения материализованного представления}
  \label{fig-task-5-mview-plan}
\end{figure}

Посмотрим на план выполнения. Видим, что очень заметно снизился cost "--- на $\sim 91$\% (70~$\to$~6).
Однако большинство статистик ухудшилось:
\begin{itemize}%
  \item \texttt{recursive calls} "--- 469~$\to$~954,
  \item \texttt{db block gets} "--- 0~$\to$~18,
  \item \texttt{consistent gets} "--- 918~$\to$~1576,
  \item \texttt{physical reads} "--- 295~$\to$~113,
  \item \texttt{redo size} "--- 0~$\to$~2940,
  \item \texttt{bytes sent via SQL*Net to client} "--- 100333~$\to$~103883,
  \item \texttt{sorts (memory)} "--- 49~$\to$~84.
\end{itemize}


% % .---|||___|||--- S U B S E C T I O N ---|||___|||---. %
% \section{Вторая оптимизация (индексно-организованная таблица)}
% % .---|||___|||--- S U B S E C T I O N ---|||___|||---. %


% \SQLplan{5-iot.txt}
% \begin{figure}[H]%
%   \caption{План выполнения оптимизированного запроса}
%   \label{fig-task-5-iot-plan}
% \end{figure}


% .---|||___|||--- S U B S E C T I O N ---|||___|||---. %
\section{Вторая оптимизация (композитный bitmap индекс)}
% .---|||___|||--- S U B S E C T I O N ---|||___|||---. %


Заметим, что в \texttt{GROUP BY} находятся 2 столбца "--- \texttt{salary} и \texttt{job\_id}, что говорит нам о том, что неплохо было бы иметь на них композитный индекс.
Создадим же композитный bitmap индекс (bitmap "--- потому что значения часто повторяются "--- иначе вряд ли есть смысл по ним группировать) "--- \firef{oh-hey-babe}.
Обратим внимание на \texttt{DESC} у \texttt{salary} "--- он используется потому, что в сортировке в исходном запросе мы сортируем по \texttt{salary~DESC} (и, вообще говоря, довольно часто в запросах \texttt{employees} сортировка по \texttt{salary} идёт именно в обратном порядке).

\begin{minted}[tabsize=2, mathescape, fontsize=\tiny]{sql}%
CREATE BITMAP INDEX employees_salary_job_idx ON hr.employees (salary DESC, job_id);
\end{minted}
\begin{figure}[H]%
  \caption{Создание композитного bitmap индекса}
  \label{oh-hey-babe}
\end{figure}

\SQLplan{5-bitmap-composite.txt}
\begin{figure}[H]%
  \caption{План выполнения оптимизированного запроса}
  \label{fig-task-5-bitmap-composite-plan}
\end{figure}

\pagebreak

Посмотрим, как эта оптимизация повлияла на план выполнения. В целом, неплохо уменьшилась стоимость (cost) "--- на $\sim 19$\% (70~$\to$~57).
Теперь посмотрим на изменение статистик (в целом, они ухудшились):
\begin{itemize}%
  \item \texttt{recursive calls} "--- 469~$\to$~506,
  \item \texttt{db block gets} "--- 0~$\to$~12,
  \item \texttt{consistent gets} "--- 918~$\to$~786,
  \item \texttt{physical reads} "--- 295~$\to$~113,
  \item \texttt{redo size} "--- 0~$\to$~2024,
  \item \texttt{sorts (memory)} "--- 49~$\to$~54.
\end{itemize}

             % Глава 2
  % \input{my_folder/chapter3}             % Глава 3
  % \input{my_folder/chapter4}             % Глава 3
  % \ContinueChapterEnd % завершить размещение глав <<подряд>>
  %!TeX root = ../My_thesis.tex
\chapter*{Заключение} \label{ch-conclusion}
\addcontentsline{toc}{chapter}{Заключение}	% в оглавление 

В данной работе было выполнено исследование различных способов оптимизации SQL-запросов на примере СУБД Oracle для 5-ти задач. Были написаны SQL-запросы для каждой задачи, были проанализированы планы выполнения, статистики и предложены по два варианта оптимизаций в каждой задаче. В ходе решения были использованы следующие способы оптимизации:
\begin{itemize}%
  \item function-based индекс,
  \item bitmap индекс,
  \item переписывание запроса,
  \item материализованное представление,
  \item индексно-организованная таблица,
  \item композитный B-tree индекс,
  \item композитный bitmap индекс.
\end{itemize}

В результате для каждого запроса удалось достигнуть снижения стоимости (cost).
На протяжении всей работы была видна тенденция к ухудшению статистик (хотя и не везде) после оптимизаций.
Объяснить это можно, вероятно, тем, что перед каждым запросом мы чистили shared pool и buffer cache, что могло негативно повлиять на запросы с индексами (по сравнению с тем, как если бы мы не чистили shared pool и buffer cache).

           % Заключение
  % %% Завершение основной части

  \input{my_folder/references}         % Список литературы
  \appendix % не редактировать / keep unmodified
\end{document}
