%!TeX root = ../My_thesis.tex


\chapter*{Описание окружения} % * не проставляет номер
\addcontentsline{toc}{chapter}{Описание окружения} % вносим в содержание


Используется база данных Oracle версии \texttt{Oracle Database 21c Express Edition Release 21.0.0.0.0 - Production Version 21.3.0.0.0}.
Схемы создавались в Portable Database~\cite{PDB} \texttt{xepdb1}, поэтому строки подключения имели вид: \texttt{CONNECT hr/PASSWORD@xepdb1}.

Стандартные схемы Oracle (\texttt{HR}, \texttt{OE} и~т.\,д.) были взяты из оффициального репозитория Oracle~\cite{SampleSchemas}.
Схема \texttt{STUDENT} была взята с учебного курса «Программирование и оптимизация баз данных»~\cite{StudentSchema}.

Перед каждым выполнением SQL-запроса выполняется очистка Buffer Cache и Shared Pool с помощью первых двух команд ниже.
После чего выполняется запрос с выводом плана.
\begin{minted}[tabsize=2, mathescape, fontsize=\footnotesize]{sql}
alter system flush shared_pool;
alter system flush buffer_cache;
set autotrace on
@task-/*НОМЕР-ЗАДАНИЯ*/.sql
set autotrace off
\end{minted}

Для схем \texttt{HR}, \texttt{OE} и \texttt{STUDENT} были сгенерированы данные (код генерации и SQL-скрипты для вставки доступны для скачивания в репозитории~\cite{Github}).
После добавления данных выполняется сбор статистики для задействованных таблиц, пример для сбора статистики таблицы \texttt{HR.EMPLOYEES} показан ниже.
\begin{minted}[tabsize=2, mathescape, fontsize=\footnotesize]{sql}
EXEC DBMS_STATS.GATHER_TABLE_STATS('HR', 'EMPLOYEES');
\end{minted}
Результаты запросов показываются для исходных данных (без сгенерированных), так как некоторые запросы выводят сотни строк со сгенерированными данными.
Все планы, включённые в данную работу, показываются для таблиц со сгенерированными данными.

Проверка наличия индексов в таблицах делалась следующим образом:
\begin{minted}[tabsize=2, mathescape, fontsize=\footnotesize]{sql}
SELECT
  CAST(ind.table_name AS VARCHAR2(36)) AS "Table",
  CAST(ind.index_name AS VARCHAR2(36)) AS "Index name",
  CAST(col.column_name AS VARCHAR2(36)) AS "Column",
  CAST(ind.index_type AS VARCHAR2(36)) AS "Index type"
FROM user_indexes ind
  JOIN user_ind_columns col
    ON ind.index_name = col.index_name
ORDER BY 1, 2, 3, 4;
\end{minted}
