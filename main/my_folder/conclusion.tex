%!TeX root = ../My_thesis.tex
\chapter*{Заключение} \label{ch-conclusion}
\addcontentsline{toc}{chapter}{Заключение}	% в оглавление 

В данной работе было выполнено исследование различных способов оптимизации SQL-запросов на примере СУБД Oracle для 5-ти задач. Были написаны SQL-запросы для каждой задачи, были проанализированы планы выполнения, статистики и предложены по два варианта оптимизаций в каждой задаче. В ходе решения были использованы следующие способы оптимизации:
\begin{itemize}%
  \item function-based индекс,
  \item bitmap индекс,
  \item переписывание запроса,
  \item материализованное представление,
  \item индексно-организованная таблица,
  \item композитный B-tree индекс,
  \item композитный bitmap индекс.
\end{itemize}

В результате для каждого запроса удалось достигнуть снижения стоимости (cost).
На протяжении всей работы была видна тенденция к ухудшению статистик (хотя и не везде) после оптимизаций.
Объяснить это можно, вероятно, тем, что перед каждым запросом мы чистили shared pool и buffer cache, что могло негативно повлиять на запросы с индексами (по сравнению с тем, как если бы мы не чистили shared pool и buffer cache).

