%!TeX root = ../My_thesis.tex
\chapter*{Введение} % * не проставляет номер
\addcontentsline{toc}{chapter}{Введение} % вносим в содержание

Оптимизация запросов "--- важная часть администрирования баз данных.
От скорости выполнения зависит скорость работы всего приложения (если база данных используется для какого-либо приложения).
К примеру, в интернет-магазине, чем дольше пользователь ждёт, пока загрузятся данные, тем выше вероятность, что он покинет сайт в пользу конкурента.
Оптимизация так же помогает избежать траты на излишнюю покупка железа, которое приобретают, чтобы ускорить работу базы данных.
Поэтому крайне важно научиться анализировать запросы на возможность их оптимизации.

Целью данной работы является практика оптимизирования запросов в СУБД Oracle.
Конкретно "--- оптимизироваться будет стоимость запросов (cost), которая вычисляется по формуле~\eqref{eq_cost}~\cite{ExplainPlan}.
\begin{equation}\label{eq_cost}%
  \operatorname{cost} = \frac{
    (\operatorname{\#SRds} \cdot \operatorname{sreadtim})
      + (\operatorname{\#MRds} \cdot \operatorname{mreadtim})
      + (\operatorname{\#cpuCycles} / \operatorname{cpuSpeed})
  }{
    \operatorname{sreadtim}
  },
\end{equation}
где
\begin{itemize}%
  \item #SRDs "--- количество одноблочных чтений.
  \item sreadtim "--- время на одно одноблочное чтение.
  \item \#MRDs "--- количество многоблочных чтений.
  \item mreadtim "--- время одно на многоблочное чтение.
  \item \#cpuCycles "--- количество циклов CPU. Включает в себя стоимость CPU-обработки (чистая стоимость CPU) и стоимость извлечения данных (стоимость buffer cache get в CPU).
  \item cpuSpeed "--- количество CPU-циклов в секунду.
\end{itemize}

Порядок выполнения работы будет следующим:
\begin{enumerate}[1.]%
  \item Написать SQL-запрос для каждой задачи.
  \item Проанализировать планы выполнения, статистики, а также сами запросы на возможные варианты оптимизаций.
  \item Использовать как минимум 2 оптимизации на каждый запрос (причём одна оптимизация может быть использована не более, чем дважды за всю работу).
  \item Оптимизация может считаться успешной при уменьшении стоимости~(cost).
\end{enumerate}
