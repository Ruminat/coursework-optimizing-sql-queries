% .---|||___|||--- S E C T I O N ---|||___|||---. %
\chapter{Задача «Процент мужщин/женщин в заданную дату (OE)»}
% .---|||___|||--- S E C T I O N ---|||___|||---. %


% .---|||___|||--- S U B S E C T I O N ---|||___|||---. %
\section{Описание задачи}
% .---|||___|||--- S U B S E C T I O N ---|||___|||---. %


Используются таблицы схемы \texttt{OE}. Вывести процентное соотношение мужчин и женщин, разместивших заказы в заданную дату. Если один и тот же человек разместил несколько заказов в заданную дату, он должен быть учтён только один раз. В результате должно быть три столбца: дата, процент мужчин и процент женщин.


% .---|||___|||--- S U B S E C T I O N ---|||___|||---. %
\section{Составленный запрос}
% .---|||___|||--- S U B S E C T I O N ---|||___|||---. %


\SQLcode{task-1.sql}
\begin{algorithm}[H]
  \caption{Запрос для задачи №1}
  \label{code-task-1}
\end{algorithm}

\begin{samepage}%
  \begin{minted}[tabsize=2, mathescape, fontsize=\tiny]{text}%
Date        Males (%) Females (%)
---------- ---------- -----------
29-06-2007         75          25
  \end{minted}
  \begin{figure}[H]%
    \caption{Результат запроса для даты «29-06-2007»}
    \label{fig-task-1-output}
  \end{figure}
\end{samepage}


\SQLplan{1-no-op.txt}
\begin{figure}[H]%
  \caption{План выполнения запроса}
  \label{fig-task-1-plan}
\end{figure}


% .---|||___|||--- S U B S E C T I O N ---|||___|||---. %
\section{Первая оптимизация (function-based индекс)}
% .---|||___|||--- S U B S E C T I O N ---|||___|||---. %


Обратим внимание на то, что в исходном запросе используется сравнение даты, использующее функцию \texttt{TRUNC}.
Создадим function-based индекс на эту функцию на столбец \texttt{order\_date}.

\begin{minted}[tabsize=2, mathescape, fontsize=\tiny]{sql}%
CREATE INDEX orders_order_date_fnidx ON oe.orders (TRUNC(order_date, 'dd'));
\end{minted}
\begin{figure}[H]%
  \caption{Создание function-based индекса}
  \label{fig-task-1-plan}
\end{figure}

\SQLplan{1-function-based.txt}
\begin{figure}[H]%
  \caption{План выполнения оптимизированного запроса}
  \label{fig-task-1-function-based-plan}
\end{figure}

Как можно увидеть по плану, cost уменьшился на $\sim 14$\% (56~$\to$~48), поменялись также статистики (не в лучшую сторону):
\begin{itemize}%
  \item \texttt{recursive calls} "--- 2732~$\to$~2992,
  \item \texttt{db block gets} "--- 0~$\to$~5,
  \item \texttt{consistent gets} "--- 3905~$\to$~4135,
  \item \texttt{physical reads} "--- 478~$\to$~491,
  \item \texttt{redo size} "--- 0~$\to$~852,
  \item \texttt{sorts (memory)} "--- 248~$\to$~271.
\end{itemize}


% .---|||___|||--- S U B S E C T I O N ---|||___|||---. %
\section{Вторая оптимизация (bitmap индекс)}
% .---|||___|||--- S U B S E C T I O N ---|||___|||---. %


Обратим также внимание на то, что столбец \texttt{gender} таблицы \texttt{customers} имеет всего 2 значения "--- «F» и «M», из чего напрашивается желание создать bitmap-индекс на этот столбец.

\begin{minted}[tabsize=2, mathescape, fontsize=\tiny]{sql}%
CREATE BITMAP INDEX customers_gender_btmidx ON oe.customers (gender);
\end{minted}
\begin{figure}[H]%
  \caption{Создание bitmap индекса}
  \label{fig-task-1-bitmap-plan}
\end{figure}

\SQLplan{1-bitmap.txt}
\begin{figure}[H]%
  \caption{План выполнения оптимизированного запроса}
  \label{fig-task-1-bitmap-plan}
\end{figure}

В плане видим, что стоимость (cost) уменьшилась на $\sim 8$\% (56~$\to$~51). Посмотрим, как поменялись статистики (в основном, в худшую сторону):
\begin{itemize}%
  \item \texttt{recursive calls} "--- 2732~$\to$~2988,
  \item \texttt{consistent gets} "--- 3905~$\to$~4087,
  \item \texttt{physical reads} "--- 478~$\to$~432,
  \item \texttt{sorts (memory)} "--- 248~$\to$~274.
\end{itemize}
