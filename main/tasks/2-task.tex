% .---|||___|||--- S E C T I O N ---|||___|||---. %
\chapter{Задача «Имена сотрудников, встречающиеся более 2-ух раз (HR)»}
% .---|||___|||--- S E C T I O N ---|||___|||---. %


% .---|||___|||--- S U B S E C T I O N ---|||___|||---. %
\section{Описание задачи}
% .---|||___|||--- S U B S E C T I O N ---|||___|||---. %


Используются таблицы схемы \texttt{HR}. Вывести имена сотрудников, встречающиеся в таблице сотрудников не менее трех раз и не являющиеся именами руководителей подразделений компании или именами непосредственных руководителей кого-либо. В результате должны быть выведены только имена сотрудников, причём каждое "--- только один раз.


% .---|||___|||--- S U B S E C T I O N ---|||___|||---. %
\section{Составленный запрос}
% .---|||___|||--- S U B S E C T I O N ---|||___|||---. %


\SQLcode{task-2.sql}
\begin{algorithm}[H]
  \caption{Запрос для задачи №2}
  \label{code-task-2}
\end{algorithm}


\begin{samepage}%
  \begin{minted}[tabsize=2, mathescape, fontsize=\tiny]{text}%
FIRST_NAME
----------
David
Peter
  \end{minted}
  \begin{figure}[H]%
    \caption{Результат запроса}
    \label{fig-task-2-output}
  \end{figure}
\end{samepage}


\SQLplan{2-no-op.txt}
\begin{figure}[H]%
  \caption{План выполнения запроса}
  \label{fig-task-2-plan}
\end{figure}


% .---|||___|||--- S U B S E C T I O N ---|||___|||---. %
\section{Первая оптимизация (переписывание запроса)}
% .---|||___|||--- S U B S E C T I O N ---|||___|||---. %


Попробуем использовать вместо \texttt{MINUS} оператор \texttt{NOT EXISTS} с подзапросом (\firef{code-task-2-rewritten}).

\SQLcode{task-2-rewritten.sql}
\begin{algorithm}[H]
  \caption{Переписанный запрос}
  \label{code-task-2-rewritten}
\end{algorithm}


\SQLplan{2-rewrite.txt}
\begin{figure}[H]%
  \caption{План выполнения переписанного запроса}
  \label{fig-task-2-rewrite-plan}
\end{figure}

Смотрим, как изменился план. Стоимость (cost) уменьшилась довольно незначительно "--- на $\sim 2$\% (107~$\to$~105).
Статистики же не изменились никоим образом.


% .---|||___|||--- S U B S E C T I O N ---|||___|||---. %
\section{Вторая оптимизация (материализованное представление)}
% .---|||___|||--- S U B S E C T I O N ---|||___|||---. %


Попробуем более масштабную оптимизацию "--- создадим материализованное представление.
Оно может очень сильно ускорить запрос.
Недостатком, тем не менее, является то, что данные в представлении нужно поддерживать (при изменении данных в исходной таблице), однако таблица \texttt{employees} маловероятно будет меняться часто (так как сотрудники в компании не приходят и уходят каждую минуту), может быть и такое, что таблица может не меняться днями. Поэтому использование материализованного представления здесь вполне оправдано.

Создание материализованного представления показано на~\firef{jojo}.

\begin{minted}[tabsize=2, mathescape, fontsize=\tiny]{sql}%
CREATE MATERIALIZED VIEW employees_names_mview
  BUILD IMMEDIATE
  REFRESH COMPLETE
  AS /* ИСХОДНЫЙ ЗАПРОС */
\end{minted}
\begin{figure}[H]%
  \caption{Создание материализованного представления \texttt{employees\_names\_mview}}
  \label{jojo}
\end{figure}

\SQLplan{2-mview.txt}
\begin{figure}[H]%
  \caption{План выполнения материализованного представления}
  \label{fig-task-2-mview-plan}
\end{figure}

Глядя на план, можно увидеть, как сильно изменился cost "--- на $\sim 97$\% (107~$\to$~3).
Некоторые статистики, тем не менее, ухудшились (кроме \texttt{physical reads} "--- они улучшились):
\begin{itemize}%
  \item recursive calls "--- 526~$\to$~934,
  \item consistent gets "--- 1189~$\to$~1460,
  \item physical reads "--- 420~$\to$~96,
  \item bytes sent via SQL*Net to client "--- 2023~$\to$~2239,
  \item sorts (memory) "--- 68~$\to$~81.
\end{itemize}
