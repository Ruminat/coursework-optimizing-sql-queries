% .---|||___|||--- S E C T I O N ---|||___|||---. %
\chapter{Задача «Вывод групп с отличниками и должниками (STUDENT)»}
% .---|||___|||--- S E C T I O N ---|||___|||---. %


% .---|||___|||--- S U B S E C T I O N ---|||___|||---. %
\section{Описание задачи}
% .---|||___|||--- S U B S E C T I O N ---|||___|||---. %


Используются таблицы схемы \texttt{STUDENT}. Создать запрос для получения информации о группах в виде, представленном в таблице\,\ref{tab-task-2-required-output}.
\begin{table}[H]%
  \centering%
  \caption{Требуемый вывод для задачи №3}
  \label{tab-task-2-required-output}
  \noindent\begin{tabularx}{\textwidth}{|C|C|C|C|C|}%
    \hline
    Группа & Кол-во студентов & Название специальности & Кол-во круглых отличников & Кол-во должников \\
    \hline
    121 &  &  &  & \\ 
    \hline
    122 &  &  &  & \\ 
    \hline
    ... &  &  &  & \\ 
    \hline
  \end{tabularx}
\end{table}%

При подсчёте отличников учесть, что пятёрка могла быть получена со второй попытки. Для всех учитывать только экзамены, предусмотренные учебным планом.


% .---|||___|||--- S U B S E C T I O N ---|||___|||---. %
\section{Составленный запрос}
% .---|||___|||--- S U B S E C T I O N ---|||___|||---. %


\SQLcode{task-3.sql}

\begin{algorithm}[H]
  \caption{Запрос для задачи №3}
  \label{code-task-3}
\end{algorithm}

\begin{minted}[tabsize=2, mathescape, fontsize=\tiny]{text}%
Группа  Кол-во студентов Кол-во студентов                    Кол-во круглых отличников Кол-во должников
------- ---------------- ----------------------------------- ------------------------- ----------------
121                    3 СИСТЕМЫ АВТОМАТИЧЕСКОГО УПРАВЛЕНИЯ                          1                1
122                    2 СИСТЕМЫ АВТОМАТИЧЕСКОГО УПРАВЛЕНИЯ                          1                0
123                    2 ЭКОНОМИКА ПРЕДПРИЯТИЙ                                       0                1
124                    2 ЭКОНОМИКА ПРЕДПРИЯТИЙ                                       0                0
\end{minted}

\begin{figure}[H]%
  \caption{Результат запроса}
  \label{fig-task-3-output}
\end{figure}


\SQLplan{3-no-op.txt}
\begin{figure}[H]%
  \caption{План выполнения запроса}
  \label{fig-task-3-plan}
\end{figure}


% .---|||___|||--- S U B S E C T I O N ---|||___|||---. %
\section{Первая оптимизация (bitmap индекс)}
% .---|||___|||--- S U B S E C T I O N ---|||___|||---. %


Заметим, что в таблице \texttt{student} столбец \texttt{group\_num} имеет тенденцию хранить похожие значения (так как для каждой группы $N$-ое количество студентов учится в ней).
Отсюда делаем вывод, что логично было бы создать bitmap~индекс на этот столбец, что мы и сделаем.

\SQLplan{3-bitmap.txt}
\begin{figure}[H]%
  \caption{План выполнения оптимизированного запроса}
  \label{fig-task-3-bitmap-plan}
\end{figure}

Смотрим план выполнения. Наблюдаем уменьшение стоимости (cost) запроса "--- на $\sim 10$\% (373~$\to$~338).
Посмотрим, как поменялись статистики (в лучшую сторону):
\begin{itemize}%
  \item \texttt{recursive calls} "--- 2471~$\to$~2446,
  \item \texttt{db block gets} "--- 23~$\to$~22,
  \item \texttt{consistent gets} "--- 4264~$\to$~3852,
  \item \texttt{physical reads} "--- 610~$\to$~414,
  \item \texttt{redo size} "--- 4052~$\to$~3832,
  \item \texttt{sorts (memory)} "--- 246~$\to$~242.
\end{itemize}


% .---|||___|||--- S U B S E C T I O N ---|||___|||---. %
\section{Вторая оптимизация (индексно-организованная таблица)}
% .---|||___|||--- S U B S E C T I O N ---|||___|||---. %


Обратим внимание на таблицу \texttt{student} "--- в ней нам нужны лишь поля \texttt{stud\_num}~и~\texttt{group\_num}. Так как остальные поля не используются, логично создать индексно-организованную таблицу \texttt{student\_lookup} (\firef{laplus-daakunesu}).
\begin{samepage}%
  \begin{minted}[tabsize=2, mathescape, fontsize=\tiny]{sql}%
DROP TABLE student_lookup PURGE;
CREATE TABLE student_lookup (
  stud_num,
  group_num,
  CONSTRAINT student_lookup_pk PRIMARY KEY (stud_num)
)
ORGANIZATION INDEX
AS
  SELECT stud_num, group_num
  FROM student.student;
  \end{minted}
  \begin{figure}[H]%
    \caption{Создание индексно-организованной таблицы \texttt{student\_lookup}}
    \label{laplus-daakunesu}
  \end{figure}
\end{samepage}

После чего в исходном запросе (\firef{code-task-3}) нам необходимо изменить все вхождения \texttt{student.student} на \texttt{student.student\_lookup} (для краткости изложения, код приводиться не будет, так как нужно поменять всего лишь 2 строки).

\SQLplan{3-iot.txt}
\begin{figure}[H]%group-num-index
  \caption{План выполнения оптимизированного запроса}
  \label{fig-task-3-iot-plan}
\end{figure}

Посмотрим на план выполнения. Видим уменьшение стоимости (cost) запроса "--- на $\sim 13$\% (373~$\to$~324).
Посмотрим, как поменялись статистики (в лучшую сторону, за исключением \texttt{consistent gets}):
\begin{itemize}%
  \item \texttt{recursive calls} "--- 2471~$\to$~2441,
  \item \texttt{consistent gets} "--- 4264~$\to$~4389,
  \item \texttt{physical reads} "--- 610~$\to$~404,
  \item \texttt{redo size} "--- 4052~$\to$~3832,
  \item \texttt{sorts (memory)} "--- 246~$\to$~242.
\end{itemize}
