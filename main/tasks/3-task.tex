% .---|||___|||--- S E C T I O N ---|||___|||---. %
\chapter{Задача «Вывод групп с отличниками и должниками (STUDENT)»}
% .---|||___|||--- S E C T I O N ---|||___|||---. %


% .---|||___|||--- S U B S E C T I O N ---|||___|||---. %
\section{Описание задачи}
% .---|||___|||--- S U B S E C T I O N ---|||___|||---. %


Используются таблицы схемы \texttt{STUDENT}. Создать запрос для получения информации о группах в виде, представленном в таблице\,\ref{tab-task-2-required-output}.
\begin{table}[H]%
  \centering%
  \caption{Требуемый вывод для задачи №3}
  \label{tab-task-2-required-output}
  \noindent\begin{tabularx}{\textwidth}{|C|C|C|C|C|}%
    \hline
    Группа & Кол-во студентов & Название специальности & Кол-во круглых отличников & Кол-во должников \\
    \hline
    121 &  &  &  & \\ 
    \hline
    122 &  &  &  & \\ 
    \hline
    ... &  &  &  & \\ 
    \hline
  \end{tabularx}
\end{table}%

При подсчёте отличников учесть, что пятёрка могла быть получена со второй попытки. Для всех учитывать только экзамены, предусмотренные учебным планом.


% .---|||___|||--- S U B S E C T I O N ---|||___|||---. %
\section{Составленный запрос}
% .---|||___|||--- S U B S E C T I O N ---|||___|||---. %


\SQLcode{task-3.sql}

\begin{algorithm}[H]
  \caption{Запрос для задачи №3}
  \label{code-task-3}
\end{algorithm}

\begin{minted}[tabsize=2, mathescape, fontsize=\tiny]{text}%
Группа  Кол-во студентов Кол-во студентов                    Кол-во круглых отличников Кол-во должников
------- ---------------- ----------------------------------- ------------------------- ----------------
121                    3 СИСТЕМЫ АВТОМАТИЧЕСКОГО УПРАВЛЕНИЯ                          1                1
122                    2 СИСТЕМЫ АВТОМАТИЧЕСКОГО УПРАВЛЕНИЯ                          1                0
123                    2 ЭКОНОМИКА ПРЕДПРИЯТИЙ                                       0                1
124                    2 ЭКОНОМИКА ПРЕДПРИЯТИЙ                                       0                0
\end{minted}

\begin{figure}[H]%
  \caption{Результат запроса}
  \label{fig-task-3-output}
\end{figure}


\SQLplan{3-no-op.txt}
\begin{figure}[H]%
  \caption{План выполнения запроса}
  \label{fig-task-3-plan}
\end{figure}


% .---|||___|||--- S U B S E C T I O N ---|||___|||---. %
\section{Первая оптимизация (bitmap индекс)}
% .---|||___|||--- S U B S E C T I O N ---|||___|||---. %


\SQLplan{3-group-num-index.txt}
\begin{figure}[H]%
  \caption{План выполнения оптимизированного запроса}
  \label{fig-task-4-b-tree-plan}
\end{figure}

