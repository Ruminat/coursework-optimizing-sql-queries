% .---|||___|||--- S E C T I O N ---|||___|||---. %
\chapter{Задача «Клиент, сделавший покупку на максимальную сумму (SH)»}
% .---|||___|||--- S E C T I O N ---|||___|||---. %


% .---|||___|||--- S U B S E C T I O N ---|||___|||---. %
\section{Описание задачи}
% .---|||___|||--- S U B S E C T I O N ---|||___|||---. %


Используются таблицы схемы \texttt{SH}. Вывести фамилию и имя клиента, сделавшего покупки через интернет (\texttt{channel\_desc = "Internet"}) или партнёров (\texttt{channel\_desc = "Partners"}) не по акции (\texttt{promo\_category = "NO PROMOTION \#"}) на максимальную сумму в заданном году.


% .---|||___|||--- S U B S E C T I O N ---|||___|||---. %
\section{Составленный запрос}
% .---|||___|||--- S U B S E C T I O N ---|||___|||---. %


\SQLcode{task-4.sql}

\begin{algorithm}[H]
  \caption{Запрос для задачи №4}
  \label{code-task-4}
\end{algorithm}

\begin{minted}[tabsize=2, mathescape, fontsize=\tiny]{text}%
Surname                                  Name                      Spent
---------------------------------------- -------------------- ----------
Bakerman                                 Marvel                 56243,93
\end{minted}

\begin{figure}[H]%
  \caption{Результат запроса для года 1998}
  \label{fig-task-4-output}
\end{figure}


\SQLplan{4-no-op.txt}
\begin{figure}[H]%
  \caption{План выполнения запроса}
  \label{fig-task-4-plan}
\end{figure}


% .---|||___|||--- S U B S E C T I O N ---|||___|||---. %
\section{Первая оптимизация (индексно-организованная таблица)}
% .---|||___|||--- S U B S E C T I O N ---|||___|||---. %


В таблице \texttt{customers} довольно большое количество столбцов (больше 10), однако в нашем запросе используются лишь 3, отсюда появляется идея создать индексно-организованную таблицу \texttt{customer\_lookup}, содержащую эти столбцы. Это мы и сделаем (\firef{pekoooooora}).

\begin{minted}[tabsize=2, mathescape, fontsize=\tiny]{sql}%
CREATE TABLE customer_lookup (
  cust_id,
  cust_last_name,
  cust_first_name,
  CONSTRAINT customers_iot_pk PRIMARY KEY (cust_id)
)
ORGANIZATION INDEX
AS
  SELECT
    cust_id,
    cust_last_name,
    cust_first_name
  FROM sh.customers;
\end{minted}
\begin{figure}[H]%
  \caption{Создание индексно-организованной таблицы \texttt{customer\_lookup}}
  \label{pekoooooora}
\end{figure}

Как и в предыдущей задаче, нам нужно всего лишь заменить все вхождения \texttt{sh.customer} на \texttt{sh.customer\_lookup}.

\SQLplan{4-iot.txt}
\begin{figure}[H]%
  \caption{План выполнения оптимизированного запроса}
  \label{fig-task-4-iot-plan}
\end{figure}

Как можно заметить на плане, стоимость запроса существенно снизилась "--- на $\sim 40$\% (1059~$\to$~635).
Посмотрим на изменение статистик (многие статистики улучшились, причём некоторые "--- довольно ощутимо, "--- хотя некоторые, всё же ухудшились, особенно \texttt{consistent gets}):
\begin{itemize}%
  \item \texttt{recursive calls} "--- 2471~$\to$~2227,
  \item \texttt{db block gets} "--- 23~$\to$~0,
  \item \texttt{consistent gets} "--- 4264~$\to$~52689,
  \item \texttt{physical reads} "--- 610~$\to$~2054,
  \item \texttt{redo size} "--- 4052~$\to$~0,
  \item \texttt{bytes sent via SQL*Net to client} "--- 38650~$\to$~735,
  \item \texttt{bytes received via SQL*Net from client} "--- 536~$\to$~52,
  \item \texttt{SQL*Net roundtrips to/from client} "--- 46~$\to$~2,
  \item \texttt{sorts (memory)} "--- 246~$\to$~197,
  \item \texttt{rows processed} "--- 671~$\to$~1.
\end{itemize}


% .---|||___|||--- S U B S E C T I O N ---|||___|||---. %
\section{Вторая оптимизация (компизитный B-tree индекс)}
% .---|||___|||--- S U B S E C T I O N ---|||___|||---. %


Есть также другой путь оптимизации 3-х столбцов из предыдущей оптимизации "--- заметим, что все эти столбцы находятся внутри \texttt{GROUP BY}, что как бы намекает нам на создание композитного B-tree индекса (B-tree потому, что значения не имеют тенденции иметь много одинаковых значений). Этим мы и займёмся "--- создадим такой индекс (\firef{baqua}).

\begin{minted}[tabsize=2, mathescape, fontsize=\tiny]{sql}%
CREATE INDEX customers_cmpidx ON sh.customers (cust_id, cust_first_name, cust_last_name);
\end{minted}
\begin{figure}[H]%
  \caption{Создание композитного индекса}
  \label{baqua}
\end{figure}

\SQLplan{4-composite.txt}
\begin{figure}[H]%
  \caption{План выполнения оптимизированного запроса}
  \label{fig-task-4-composite-plan}
\end{figure}

Теперь посмотрим, как изменился cost. Улучшение поменьше, чем при первой оптимизации "--- на $\sim 34$\% (1059~$\to$~700), "--- однако плюсом этой оптимизации является то, что нам не нужно поддерживать отдельную таблицу \texttt{sh.customer\_lookup}, хотя и нужно поддерживать индекс, при этом прошлая оптимизация работает лишь для \texttt{sh.customer\_lookup}, текущая "--- для изначальной таблицы.
Посмотрим на статистики (здесь, по сравнению с прошлой оптимизацией, не произошло ухудшения статистик, однако и улучшения не столь значительны):
\begin{itemize}%
  \item \texttt{recursive calls} "--- 4230~$\to$~4217,
  \item \texttt{consistent gets} "--- 10016~$\to$~8701,
  \item \texttt{physical reads} "--- 3582~$\to$~2286,
  \item \texttt{sorts (memory)} "--- 374~$\to$~370.
\end{itemize}
