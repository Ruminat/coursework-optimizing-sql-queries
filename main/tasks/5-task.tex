% .---|||___|||--- S E C T I O N ---|||___|||---. %
\chapter{Задача «Список сотрудников по должностям и зарплатам (HR)»}
% .---|||___|||--- S E C T I O N ---|||___|||---. %


% .---|||___|||--- S U B S E C T I O N ---|||___|||---. %
\section{Описание задачи}
% .---|||___|||--- S U B S E C T I O N ---|||___|||---. %


Используются таблицы схемы \texttt{HR}. Одной командой \texttt{SELECT} вывести список сотрудников компании, имеющих коллег с таким же идентификатором должности и окладом. Если некоторый идентификатор должности и размер оклада имеет один единственный сотрудник, то сведения о нём в результат попадать не должны.

В результат вывести:
\begin{enumerate}[1.]%
  \item идентификатор должности;
  \item размер оклада;
  \item список фамилий сотрудников, имеющих данный идентификатор должности и данный оклад.
\end{enumerate}

Фамилии в списке должны быть:
\begin{enumerate}[a.]%
  \item упорядочены по алфавиту (по возрастанию),
  \item разделены символами \texttt{', '} («запятая» и «пробел»),
  \item перед первой фамилией не должно быть символов-разделителей,
  \item после последней фамилии символов-разделителей быть не должно.
\end{enumerate}

Результат упорядочить:
\begin{enumerate}[1.]%
  \item по размеру оклада (по убыванию),
  \item по идентификатору должности (по возрастанию).
\end{enumerate}


% .---|||___|||--- S U B S E C T I O N ---|||___|||---. %
\section{Составленный запрос}
% .---|||___|||--- S U B S E C T I O N ---|||___|||---. %


\SQLcode{task-5.sql}

\begin{algorithm}[H]
  \caption{Запрос для задачи №5}
  \label{code-task-5}
\end{algorithm}

\begin{minted}[tabsize=2, mathescape, fontsize=\tiny]{text}%
Job ID         Salary Surnames
---------- ---------- ------------------------------
AD_VP           17000 De Haan, Kochhar
SA_REP          10000 Bloom, King, Tucker
SA_REP           9500 Bernstein, Greene, Sully
SA_REP           9000 Hall, McEwen
SA_REP           8000 Olsen, Smith
SA_REP           7500 Cambrault, Doran
SA_REP           7000 Grant, Sewall, Tuvault
SA_REP           6200 Banda, Johnson
IT_PROG          4800 Austin, Pataballa
ST_CLERK         3300 Bissot, Mallin
SH_CLERK         3200 McCain, Taylor

Job ID         Salary Surnames
---------- ---------- ------------------------------
ST_CLERK         3200 Nayer, Stiles
SH_CLERK         3100 Fleaur, Walsh
SH_CLERK         3000 Cabrio, Feeney
SH_CLERK         2800 Geoni, Jones
ST_CLERK         2700 Mikkilineni, Seo
SH_CLERK         2600 Grant, OConnell
SH_CLERK         2500 Perkins, Sullivan
ST_CLERK         2500 Marlow, Patel, Vargas
ST_CLERK         2400 Gee, Landry
ST_CLERK         2200 Markle, Philtanker
\end{minted}
\begin{figure}[H]%
  \caption{Результат запроса}
  \label{fig-task-5-output}
\end{figure}


\SQLplan{5-no-op.txt}
\begin{figure}[H]%
  \caption{План выполнения запроса}
  \label{fig-task-5-plan}
\end{figure}


% .---|||___|||--- S U B S E C T I O N ---|||___|||---. %
\section{Первая оптимизация (материализованное представление)}
% .---|||___|||--- S U B S E C T I O N ---|||___|||---. %


\SQLplan{5-mview.txt}
\begin{figure}[H]%
  \caption{План выполнения материализованного представления}
  \label{fig-task-5-mview-plan}
\end{figure}


% % .---|||___|||--- S U B S E C T I O N ---|||___|||---. %
% \section{Вторая оптимизация (индексно-организованная таблица)}
% % .---|||___|||--- S U B S E C T I O N ---|||___|||---. %


% \SQLplan{5-iot.txt}
% \begin{figure}[H]%
%   \caption{План выполнения оптимизированного запроса}
%   \label{fig-task-5-iot-plan}
% \end{figure}


% .---|||___|||--- S U B S E C T I O N ---|||___|||---. %
\section{Вторая оптимизация (композитный bitmap индекс)}
% .---|||___|||--- S U B S E C T I O N ---|||___|||---. %


\SQLplan{5-bitmap-composite.txt}
\begin{figure}[H]%
  \caption{План выполнения оптимизированного запроса}
  \label{fig-task-5-bitmap-composite-plan}
\end{figure}
