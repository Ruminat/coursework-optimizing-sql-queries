% .---|||___|||--- S E C T I O N ---|||___|||---. %
\chapter{Задача «Имена сотрудников, встречающиеся более 2-ух раз (HR)»}
% .---|||___|||--- S E C T I O N ---|||___|||---. %


% .---|||___|||--- S U B S E C T I O N ---|||___|||---. %
\section{Описание задачи}
% .---|||___|||--- S U B S E C T I O N ---|||___|||---. %


Используются таблицы схемы \texttt{HR}. Вывести имена сотрудников, встречающиеся в таблице сотрудников не менее трех раз и не являющиеся именами руководителей подразделений компании или именами непосредственных руководителей кого-либо. В результате должны быть выведены только имена сотрудников, причём каждое "--- только один раз.


% .---|||___|||--- S U B S E C T I O N ---|||___|||---. %
\section{Составленный запрос}
% .---|||___|||--- S U B S E C T I O N ---|||___|||---. %


\SQLcode{task-2.sql}
\begin{algorithm}[H]
  \caption{Запрос для задачи №2}
  \label{code-task-2}
\end{algorithm}


\begin{samepage}%
  \begin{minted}[tabsize=2, mathescape, fontsize=\tiny]{text}%
FIRST_NAME
----------
David
Peter
  \end{minted}
  \begin{figure}[H]%
    \caption{Результат запроса}
    \label{fig-task-2-output}
  \end{figure}
\end{samepage}


\SQLplan{2-no-op.txt}
\begin{figure}[H]%
  \caption{План выполнения запроса}
  \label{fig-task-2-plan}
\end{figure}


% .---|||___|||--- S U B S E C T I O N ---|||___|||---. %
\section{Первая оптимизация (переписывание запроса)}
% .---|||___|||--- S U B S E C T I O N ---|||___|||---. %


\SQLcode{task-2-rewritten.sql}
\begin{algorithm}[H]
  \caption{Переписанный запрос}
  \label{code-task-2-rewritten}
\end{algorithm}


\SQLplan{2-rewrite.txt}
\begin{figure}[H]%
  \caption{План выполнения переписанного запроса}
  \label{fig-task-2-rewrite-plan}
\end{figure}


% .---|||___|||--- S U B S E C T I O N ---|||___|||---. %
\section{Вторая оптимизация (материализованное представление)}
% .---|||___|||--- S U B S E C T I O N ---|||___|||---. %


% \SQLcode{task-2-.sql}
% \begin{algorithm}[H]
%   \caption{Переписанный запрос}
%   \label{code-task-2-rewritten}
% \end{algorithm}


\SQLplan{2-mview.txt}
\begin{figure}[H]%
  \caption{План выполнения материализованного представления}
  \label{fig-task-2-mview-plan}
\end{figure}

